\documentclass[11pt,a4paper]{article}

\usepackage[
    a4paper,
    left=15mm,
    right=15mm,
    top=30mm,
    bottom=25mm,
    headheight=25mm
]{geometry}
\usepackage{graphicx}
\usepackage{polski}
\usepackage[utf8]{inputenc}
\usepackage{enumerate}
\usepackage{comment}
\usepackage{fancyhdr}
\usepackage{hyperref}
\usepackage{indentfirst}
\usepackage{multirow}
\usepackage{multicol}
\usepackage{float}
\usepackage{amsmath}
\usepackage{fancyhdr}
\usepackage{wrapfig}
\usepackage{layout}
\usepackage{textcomp}
\usepackage[center]{caption}
\usepackage{subcaption}
\usepackage{siunitx}
\usepackage[mathscr]{euscript}

\sisetup{output-exponent-marker=\ensuremath{\mathrm{e}}}

\renewcommand{\baselinestretch}{1.18}
\renewcommand\thesubfigure{\roman{subfigure}}

\pagestyle{fancy}
\fancyhead[L]{
    \includegraphics[scale=0.16]{./res/agh_logo_text_asym.jpg}
}
\fancyhead[R]{
    Zadanie obiczeniowe - Potencjał grawitacyjny.
    \\
    Łukasz Dragon
}

\begin{document}
\section{Opis problemu}
Celem obliczeń jest znalezienie rozwiązania dla:
\begin{equation}
    \begin{split}
        \frac{d^2 \Phi}{dx^2} &= 4\pi G \rho(x)
        \\
        &\Omega = (0; 3)
        \\
        &\Phi(0) = 5
        \\
        &\Phi(3) = 4
        \\
        \rho(x) = &
        \begin{cases}
            0 & x \in [0; 1] \\
            1 & x \in (1; 2] \\
            0 & x \in (2; 3]
        \end{cases}
    \end{split}
\end{equation}
korzystając z metody elementów skończonych.

\section{Obliczenia}
\subsection{Wyprowadzenie sformułowania wariacyjnego}
Niech $V$ oznacza przestrzeń funkcji zerujących
się na brzegach $\Omega$.
Szukamy funkcji $\Phi$ spełniającej:
\begin{equation}
    \forall v \in V \quad \int_{\Omega}^{} \frac{d^2 \Phi}{dx^2}v dx 
    = \int_{\Omega}^{} 4\pi G \rho v dx \quad (v(0) = 0, v(3) = 0)
\end{equation}
Przekształcając równanie:
\begin{equation}
    \begin{split}
        \int_{0}^{3} \Phi''vdx = 4\pi G\int_{0}^{3}\rho vdx 
        &\implies
        \\
        [\Phi'v]^3_0 - \int_{0}^{3} \Phi'v'dx = 4\pi G \int_{0}^{3}\rho vdx
        &\implies
        \\
        (\Phi(3)'\underbrace{v(3)}_{=0} - \Phi(0)'\underbrace{v(0)}_{=0}) 
        - \int_{0}^{3} \Phi'v'dx = 4\pi G \int_{0}^{3}\rho vdx
        &\implies
        \\
        - \int_{0}^{3} \Phi'v'dx = 4\pi G \int_{0}^{3}\rho vdx
    \end{split}
\end{equation}

Wprowadźmy podstawienie: $\Phi = w + \widetilde{\Phi}$, gdzie:
$w(0) = 0, w(3) = 0$ oraz $\widetilde{\Phi}(0) = 5, \widetilde{\Phi}(3) = 4$.
Wybierzmy $\widetilde{\Phi} = -\frac{1}{3}x + 5$.
Podstawiając do równania:
\begin{equation}
    \begin{split}
        -\int_{0}^{3} (w + \widetilde{\Phi})'v'dx = 4\pi G \int_{0}^{3}\rho v dx
        &\implies
        \\
        -\int_{0}^{3}w'v'dx - \int_{0}^{3}\widetilde{\Phi}'v'dx = 4\pi G \int_{0}^{3}\rho v dx
        &\implies
        \\
        -\int_{0}^{3}w'v'dx - \int_{0}^{3}(-\frac{1}{3}x + 5)'v'dx = 4\pi G \int_{0}^{3}\rho v dx
        &\implies
        \\
        -\int_{0}^{3}w'v'dx + \frac{1}{3} \int_{0}^{3}v'dx = 4\pi G \int_{0}^{3}\rho v dx
        &\implies
        \\
        -\int_{0}^{3}w'v'dx + \frac{1}{3} \underbrace{[v]^3_0}_{=0} = 4\pi G \int_{0}^{3}\rho v dx
        &\implies
        \\
        -\int_{0}^{3}w'v'dx = 4\pi G \int_{0}^{3}\rho v dx
    \end{split}
\end{equation}
Wprowadźmy oznaczenia $B(w, v) = -\int_{0}^{3}w'v'dx$ oraz
$L(v) = 4\pi G \int_{0}^{3}\rho v dx$, gdzie $B(w, v)$ jest
\textit{funkcjonałem biliniowym} i $L(v)$ 
jest \textit{funkcjonałem liniowym}.

Równanie możemy więc zapisać w postaci:
\begin{equation}
    B(w, v) = L(v)
\end{equation}

\subsection{Aproksymacja metodą elementów skończonych}
Podzielmy przedział $[0; 3]$ na $n$ części równej długości:
\begin{equation}
    \begin{split}
        h = \frac{3}{n}
        \\
        x_0 = 0, x_n = 3,  x_i = ih, 
        \\
        \implies
        \\
        x_{i - 1} = x_i - h, x_{i + 1} = x_i + h 
    \end{split}
\end{equation}

Na ich podstawie wybierzmy funkcje bazowe:
\begin{equation}
    \begin{split}
        \forall i = 1, 2, ... n - 1 \quad e_i(x) &= 
        \begin{cases}
            \frac{x - x_{i - 1}}{x_i - x_{i - 1}} & x \in (x_{i - 1}, x_i) \\
            \frac{x_{i + 1} - x}{x_{i + 1} - x_i} & x \in [x_i, x_{i + 1}) \\
            0 & \text{inaczej}
        \end{cases}
        \\
        &= 
        \begin{cases}
            \frac{x}{h} - i + 1 & x \in (x_{i - 1}, x_i) \\
            -\frac{x}{h} + i + 1 & x \in [x_i, x_{i + 1}) \\
            0 & \text{inaczej}
        \end{cases}
        \\
        \implies
        \\
        e'_i(x) &= 
        \begin{cases}
            \frac{1}{h} & x \in (x_{i - 1}, x_i) \\
            -\frac{1}{h} & x \in [x_i, x_{i + 1}) \\
            0 & \text{inaczej}
        \end{cases}
    \end{split}
\end{equation}

Rozważamy teraz problem przybliżony:
\begin{equation}
    \begin{split}
        w(x) \approx w_h(x) = \sum_{i = 1}^{n - 1} w_i e_i(x)
        \\
        B(\sum_{i = 1}^{n - 1} w_i e_i, v) = L(v).
    \end{split}
\end{equation}

Równanie musi być prawdziwe dla dowolnej funkcji testowej $v$,
zatem możemy przyjąć:
\begin{equation}
    \forall i = 1, 2, ..., n - 1 \quad B(\sum_{j = 1}^{n - 1} w_j e_j, e_i) = L(e_i).
\end{equation}

Korzystając z faktu, że $B$ jest \textit{funkcjonałem biliniowym} otrzymujemy układ równań:
\begin{equation}
    \forall i = 1, 2, ..., n - 1 \quad \sum_{j = 1}^{n - 1} w_j B(e_j, e_i) = L(e_i).
\end{equation}

Zapiszmy otrzymany układ równań liniowych w postaci macierzowej:
\begin{equation}
    \begin{bmatrix}
        B(e_1, e_1) & B(e_2, e_1) & \dots & B(e_{n - 1}, e_1) \\
        B(e_1, e_2) & B(e_2, e_2) & \dots & B(e_{n - 1}, e_2) \\
        \vdots & \vdots & \ddots & \vdots \\
        B(e_1, e_{n - 1}) & B(e_2, e_{n - 1}) & \dots & B(e_{n - 1}, e_{n - 1}) \\
    \end{bmatrix}
    \begin{bmatrix}
        w_1 \\ w_2 \\ \vdots \\ w_{n - 1}
    \end{bmatrix}
    =
    \begin{bmatrix}
        L(e_1) \\ L(e_2) \\ \vdots \\ L(e_{n - 1})
    \end{bmatrix}
\end{equation}

Rozpiszmy teraz funkcjonały $B$ i $L$:
\begin{equation}
    \begin{split}
        B(e_j, e_i) = \int_{0}^{3} e'_j e'_i dx = \int_{X_{i,j}} e'_j e'_i dx\\
        \text{gdzie} \quad X_{i, j} = 
        \begin{cases}
            (x_{i - 1}, x_{i + 1}) & i = j \\
            (x_i, x_j) & i = j - 1 \\
            (x_j, x_i) & j = i - 1 \\
            \emptyset & inaczej
        \end{cases}
    \end{split}
\end{equation}
\begin{equation}
    L(e_i) = 4\pi G \int_{0}^{3}\rho e_i dx 
    = 0 + 4\pi G \int_{1}^{2} e_i dx + 0
    = 4\pi G \int_{max\{1, x_{i - 1}\}}^{min\{2, x_{i + 1}\}} e_i dx
\end{equation}



Dzięki otrzymanym wzorom dla $B$ i $L$ możemy zapisać nasze równanie
macierzowe jako:
\begin{equation}
    \begin{bmatrix}
        \int_{x_0}^{x_2} (e'_1)^2 dx & \int_{x_1}^{x_2} e'_2 e'_1 dx & 0 & \dots & 0 \\
        \int_{x_1}^{x_2} e'_1 e'_2 dx & \int_{x_1}^{x_3} (e'_2)^2 dx & \int_{x_2}^{x_3} e'_3 e'_2 & \dots & 0\\
        0 & \int_{x_2}^{x_3} e'_2 e'_3 dx & \int_{x_2}^{x_4} (e'_3)^2 dx & \dots & 0\\
        \vdots & \vdots & \vdots & \ddots & \vdots \\
        0 & 0 & 0 & \dots & \int_{x_{n - 2}}^{x_{n}} (e'_{n - 1})^2 dx
    \end{bmatrix}
    \begin{bmatrix}
        w_1 \\ w_2 \\ w_3 \\ \vdots \\ w_{n - 1}
    \end{bmatrix}
    = 4 \pi G
    \begin{bmatrix}
        \int_{max\{1, x_0\}}^{min\{2, x_2\}} e_1 dx \\
        \int_{max\{1, x_1\}}^{min\{2, x_3\}} e_2 dx \\
        \int_{max\{1, x_2\}}^{min\{2, x_4\}} e_3 dx \\
        \vdots \\
        \int_{max\{1, x_{n - 2}\}}^{min\{2, x_n\}} e_{n - 1} dx \\
    \end{bmatrix}
\end{equation}

Po rozwiązaniu równania jesteśmy w stanie otrzymać
aproksymację szukanej przez nas funkcji $\Phi$:
\begin{equation}
    \Phi(x) \approx w_h + \widetilde{\Phi}
    = -\frac{1}{3} x + 5 + \sum_{i = 1}^{n - 1} w_i e_i(x)
\end{equation}

\subsection{Kwadratura Gaussa}
W celu aproksymacji wartości całki oznaczonej
skorzystamy z kwadratury Gaussa o dwóch punktach.
Wzór jest następujący:
\begin{equation}
    \int_{a}^{b} f(x) dx \approx
    \frac{b - a}{2}[f(\frac{b - a}{2}\cdot \frac{1}{\sqrt{3}} + \frac{a + b}{2})
    + f(\frac{b - a}{2}\cdot \frac{-1}{\sqrt{3}} + \frac{a + b}{2})]
\end{equation}

\end{document}